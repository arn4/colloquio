
\begin{frame}[noframenumbering]
  \frametitle{Derivation of RBM probability distribution}
  \begin{alertblock}{Cliques}
    The only cliques of an RBM are all the pairs \((v_i,h_j)\).
  \end{alertblock}
  It follows from \emph{Hammersley-Clifford Theorem}\cite{fischer2012introduction} that 
  \[\prob{\vec{v}, \vec{h}} = \frac{1}{Z} \prod_{i=1,j=1}^{m,n} \psi_{i,j}(v_i, h_j),\]
  where
  \[Z = \sum_{\vec{v}, \vec{h}}\prod_{i=1,j=1}^{m,n} \psi_{i,j}(v_i, h_j)\]
  \pause
  \[
  \prob{\vec{v}, \vec{h}} 
  = \frac{1}{Z} \exp\left(\sum_{i=1,j=1}^{m,n}\log\left(\psi_{i,j}(v_i, h_j)\right)\right)
  = \frac{1}{Z} \exp\left(-E(\vec{v}, \vec{h})\right)
  \]
\end{frame}

\begin{frame}[noframenumbering]
  \frametitle{Derivation of RBM probability distribution}
  \(E(\vec{v}, \vec{h})\) must be of the form
  \[
  E(\vec{v}, \vec{h}) = \sum_{i}^m f_i{(v_i)} + \sum_{j}^n g_j{(h_j)} +
  \sum_{i=1,j=1}^{m,n} I_{i,j}{(v_i,h_j)}
  \]
  \pause
  If only polynomial energy functions are allowed:
  \[\text{Since}\quad v_i^k =  v_i \text{ and } h_j^k=h_j \qquad \forall k \in \mathds{N},\]
  \center{most general energy is}
  \[
  E(\vec{v}, \vec{h}) = -\sum_{i}^m b_i v_i - \sum_{j}^n c_j h_j -
  \sum_{i=1,j=1}^{m,n} v_i w_{i,j} h_j
  \]
\end{frame}

\begin{frame}
  \frametitle{Pseudo-likelihood}
  
  \begin{alertblock}{Problem!}
    Likelihood is not computable in practice! We can't compute \(Z\)
  \end{alertblock}
  \pause
  \[
  \vec{\bar{v}} \xrightarrow{\text{MC sample}} \vec{\bar{h}} \longrightarrow \log\likelihood{\vec{\theta}}{\vec{\bar{v}}} \approx \log{\condprob{\vec{\bar{v}}}{\vec{\bar{h}}}}
  \]
  \(\condprob{\vec{\bar{v}}}{\vec{\bar{h}}}\) can be easily computed analytically.
\end{frame}