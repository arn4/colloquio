
\section{Extended Mean Field Training}
All classical training algorithms are clever way of estimating expected values
over the distribution of the RBM. The algorithms are intrinsically stochastic
because random samples generation is used in all of them. We analyze now a completely
different approach by Gabriè~\cite{gabrie18training}, inspired by mean-field
theory of the  Ising model in Statistical Mechanics. The key idea is notice that the
log-likelihood contains a term that is the free energy of the model. Moreover the
algorithm is deterministic, since no random sample is needed.

We present a pure mathematical derivation, highlighting the physics that inspired the
steps in {\bf \color{physics-blue}blue~boxes}. We deal only with classical binary RBMs,
but the method can be generalized. We introduce also a few more notation in
{\bf \color{pink-notation}pink~boxes}.

% There are many symbols often used in the calculations.
% To amke typing faster I define few more shorcuts
\newcommand{\FEn}[2]{F{\left[#1;#2\right]}}
\newcommand{\Gam}[2]{\Gamma{\left[#1;#2\right]}}
\newcommand{\DistributionMean}[4]{\left\langle#1\right\rangle_{#4{\left[#2;#3\right]}}}
\newcommand{\Fmean}[3]{\DistributionMean{#1}{#2}{#3}{f}}
\newcommand{\Gmean}[3]{\DistributionMean{#1}{#2}{#3}{\gamma}}
\newcommand{\qhat}[1]{\vec{\hat{q}}{(#1)}}
\newcommand{\Lambd}[2]{\vec{\lambda}{\left[#1;#2\right]}}

% Really short shortcuts
% I know that it's orrible what I'm doing, but I need to not die texxing all the calculations
\newcommand{\lgt}[2]{\lambda_t{\left[#1;#2\right]}}
\newcommand{\lvi}[2]{\lambda_i^v{\left[#1;#2\right]}}
\newcommand{\lhj}[2]{\lambda_j^h{\left[#1;#2\right]}}
\newcommand{\lgtz}[1]{\lgt{#1}{0}}
\newcommand{\lviz}[1]{\lvi{#1}{0}}
\newcommand{\lhjz}[1]{\lhj{#1}{0}}
\newcommand{\lgtm}[1]{\lgt{\vec{m}}{#1}}
\newcommand{\lvim}[1]{\lvi{\vec{m}}{#1}}
\newcommand{\lhjm}[1]{\lhj{\vec{m}}{#1}}
\newcommand{\lgtmz}{\lgt{\vec{m}}{0}}
\newcommand{\lvimz}{\lvi{i\vec{m}}{0}}
\newcommand{\lhjmz}{\lhj{\vec{m}}{0}}

\newcommand{\mgt}{m_t}
\newcommand{\mvi}{m_i^v}
\newcommand{\mhj}{m_j^h}
\newcommand{\mvisq}{{\left(m_i^v\right)}^2}
\newcommand{\mhjsq}{{\left(m_j^h\right)}^2}

\newcommand{\hmgt}{\hat{m}_t}
\newcommand{\hmvi}{\hat{m}_i^v}
\newcommand{\hmhj}{\hat{m}_j^h}
\newcommand{\hmvisq}{{\left(\hat{m}_i^v\right)}^2}
\newcommand{\hmhjsq}{{\left(\hat{m}_j^h\right)}^2}

\newcommand{\sumt}{\sum_{t=1}^{m+n}}
\newcommand{\prodt}{\prod_{t=1}^{m+n}}

\newcommand{\sums}{\sum_{\vec{s}}}
\newcommand{\sumst}{\sum_{s_t \in \{0,1\}}}

\newcommand{\GammaExp}[1]{\exp{\left(-\beta E(\vec{s#1}) +    
                                                \Lambd{\vec{m}}{\beta}\cdot(\vec{s#1}-\vec{m})\right)}}


\subsection[Introducing F]{Introducing \(F\)}
\NotationBox{\vec{s}}{
  During the derivation we use the more concise notation
  \[
    \vec{s} \coloneqq (\vec{v},\vec{h}),
  \]
  since many calculations we do hold both for hidden and visible layer.
  We use the index \(t\) for sums over \(\vec{s}\) components. In example
  \[
    \left|\vec{s}\right|^2 = \left|\vec{v}\right|^2 + \left|\vec{h}\right|^2
                           = \sum_{i=1}^{m} v_i^2 + \sum_{j=1}^{n} h_j^2
                           = \sumt s_t^2.
  \]
  We can also define vector indexed as \(\vec{s}\). In example a fictitious vector \(\vec{z}\)
  \[
    \vec{z} \coloneqq (\vec{z}^v,\vec{z}^h) \qquad
    \vec{s}\cdot\vec{z} = \vec{v}\cdot\vec{z}^v + \vec{h}\cdot\vec{z}^h
                        = \sum_{i=1}^{m} v_i z^v_i + \sum_{j=1}^{n} h_j z^h_j
                        = \sumt s_t z_t,
  \]
  and now should be completely clear how this notation works.
}

We recall from previous section that the log-likelihood for an RBM is given by
\begin{equation} \label{eq:log-like-begiing-EMF}
  \log\likelihood{\vec{\theta}}{\vec{\bar{v}}}
    = \log\left(\sum_{\vec{h}} \exp\left[-E(\vec{\bar{v}},\vec{h})\right]\right)
      - \log\left(\sums \exp\left[-E(\vec{s})\right]\right)
\end{equation}
Once took the gradient the first term of this expression can be computed analytically,
as we already seen. The second term leads to the expected values instead, so let's
focus on on it. Let's define
\begin{equation} \label{eq:def-F}
  \FEn{\vec{q}}{\beta} \coloneqq - \frac{1}{\beta}
                           \log{\left[\sums \exp{\left(-\beta E(\vec{s}) + \beta \vec{q}\cdot\vec{s}\right)}\right]},
\end{equation}
so that the second term in equation~\eqref{eq:log-like-begiing-EMF} is exactly
\(\FEn{\vec{0}}{1}\).
Using these considerations the gradient of log-likelihood can be expressed in function of
\(F\). In example equation~\eqref{eq:logL-gradient-w-RBM} becomes
\[
  \ParDer{\log\likelihood{\vec{\theta}}{\vec{\bar{v}}}}{w_{i,j}} =
    \bar{v}_i \sigmoid{\sum_{i'=1}^m w_{i',j}\bar{v}_{i'}+c_j}
    -\ParDer{\left(\FEn{\vec{0}}{1}\right)}{w_{i,j}}.
\]
For now we've only lay the foundation of the computation.

\PhysicsBox{What are \(F\) and \(\vec{q}\) in  physics?}{
  \(F\) is the Helmholtz free energy; \(\vec{q}\) the external magnetic field.
  %TODO: explain much more. Why are not b and c the external fields?
}

\subsection[Computing F: \textGamma and A]{Computinng \(F\): \(\Gamma\) and \(A\)}
It's clear looking at equation~\eqref{eq:def-F} that \(F\) is at least \(C^\infty\)
in \(\vec{q}\). Moreover it's a concave function, so \(-F\) is a convex and can be
defined its Legendre~Transformation\footnote{The definition of Legendre Trasformation
  used here it's more general than the one usually taught during courses. The best
  source I found at the moment is Wikipedia.}
\begin{equation}
  \Gam{\vec{m}}{\beta} \coloneqq \sup_{\vec{q}}{\left\{\vec{q}\cdot\vec{m}
                            - \left(-\FEn{\vec{q}}{\beta}\right)\right\}}
                       = \sup_{\vec{q}}{\left\{\vec{q}\cdot\vec{m}
                            + \FEn{\vec{q}}{\beta}\right\}}
                       = \qhat{\vec{m}}\cdot\vec{m} + \FEn{\qhat{\vec{m}}}{\beta}
\end{equation}
where
\[
  \qhat{\vec{m}} \coloneqq
  \argsup_{\vec{q}}{\left\{\vec{q}\cdot\vec{m} + \FEn{\vec{q}}{\beta}\right\}}.
\]
Since Legendre Transformation is involutive we have
\[
  -\FEn{\vec{q}}{\beta} = \sup_{\vec{m}}{\left\{\vec{q}\cdot\vec{m} - \Gam{\vec{m}}{\beta}\right\}},
\]
from which it's easy to see
\begin{equation*}
  -\beta \FEn{\vec{0}}{\beta} = -\beta \sup_{\vec{m}}{\left\{\vec{0}\cdot\vec{m} -
                                                             \Gam{\vec{m}}{\beta}\right\}}
                              = -\beta \sup_{\vec{m}}{\left\{-\Gam{\vec{m}}{\beta}\right\}}
                              = -\beta \inf_{\vec{m}}{\left\{\Gam{\vec{m}}{\beta}\right\}}
                              = -\beta \Gam{\vec{\hat{m}}}{\beta},
\end{equation*}
where 
\[
  \vec{\hat{m}} \coloneqq
  \arginf_{\vec{m}}{\left\{\Gam{\vec{m}}{\beta}\right\}}.
\]
Note in particular that the definitions of \(\qhat{\vec{m}}\) and \(\vec{\hat{m}}\) are quite
different. Our attention now shifts to \(\Gamma\) since if we compute and minimize it we
can get easily \(\FEn{\vec{0}}{1}\).

We prefer deal with a different quantity rather than \(\Gamma\), so we define
\[
  A{\left[\vec{m};\beta\right]} \coloneqq -\beta \Gam{\vec{m}}{\beta}.
\]

\PhysicsBox{What are \(\Gamma\) and \(\vec{m}\) in  physics?}{
  \(\Gamma\) is the Gibbs free energy; \(\vec{m}\) is the magnetization.
  %TODO: explain much more.
}

\subsection{Computing A}
\NotationBox{\Fmean{\cdot}{\vec{q}}{\beta}}{
  If we call
  \[
    f{\left[\vec{q};\beta\right](\vec{s})} \coloneqq \exp{\left(-\beta E(\vec{s}) 
                                            + \beta \vec{q}\cdot\vec{s}\right)},
  \]
  we can write \(F\) as 
  \[
    \FEn{\vec{q}}{\beta} = -\frac{1}{\beta} \log{\left[\sums       
                                          f{\left[\vec{q};\beta\right](\vec{s})}\right]}.
  \]
  
  Given a generic operator \(O{(\vec{s})}\) it's well defined
  \[
    \Fmean{O}{\vec{q}}{\beta} =
      \frac{\sums O{(\vec{s})}f{\left[\vec{q};\beta\right](\vec{s})}}
           {\sum_{\vec{s'}}f{\left[\vec{q};\beta\right](\vec{s'})}}.
  \]
  \PhysicsBox{What is this in physical term?}{
    It's just the mean value  over the ensemble of the operator \(O\), written as
    function of independent variables \(\vec{q}\). 
  }
}
We would like to find a nicer expression for
\begin{equation} \label{eq:expression-for-A}
   A{\left[\vec{m};\beta\right]} =
    -\beta \left(\qhat{\vec{m}}\cdot\vec{m} + \FEn{\qhat{\vec{m}}}{\beta}\right).
\end{equation}
We can do the following observations on the expression 
\(-\beta \left(\vec{q}\cdot\vec{m} + \FEn{\vec{q}}{\beta}\right)\):
\begin{enumerate}[i]
  \item \(F\) is a smooth function of \(\vec{q}\), so the expression is smooth too and its
    extremes points are stationary;
  \item \(F\) is concave, so also the expression it is, since the term \(\qhat{\vec{m}}\cdot\vec{m}\)
    doesn't affect the convexity;
  \item given the previous two observation we can say that the expression has at most 1 stationary
    point and if it exist it must be a maximum.
\end{enumerate}
We can now write an equation for \(\qhat{\vec{m}}\)
\begin{align*}
  \vec{\nabla}_{\vec{q}} \Big(\vec{q}\cdot\vec{m} + \FEn{\vec{q}}{\beta}\Big)
    \bigg|_{\vec{q}=\qhat{\vec{m}}}&=0 \\
  \vec{m} + \left.\ParDer{F}{\vec{q}}\right|_{\vec{q}=\qhat{\vec{m}}}&=0 \\
  \vec{m} - \Fmean{\vec{s}}{\qhat{\vec{m}}}{\beta} = 0.
\end{align*}
In other words we can say that the function \(\qhat{\vec{m}}\) is the inverse of
magnetization function
\begin{equation}
  \label{eq:mean-magnetization}
  \vec{\bar{m}}{[\vec{q}]} \coloneqq \Fmean{\vec{s}}{\vec{q}}{\beta}.
\end{equation}
Starting from equation~\eqref{eq:expression-for-A} we can write
\begin{align}
  \nonumber
   A{\left[\vec{m};\beta\right]} &=
    \log{\left[\exp{\left(-\beta \vec{m}\cdot\qhat{\vec{m}}\right)}\right]}
      + \log{\left[\sums \exp{\left(-\beta E(\vec{s}) +  
                                        \beta\qhat{\vec{m}}\cdot\vec{s}\right)}\right]} \\
   \nonumber
   &= \log{\left[\exp{\left(-\beta \vec{m}\cdot\qhat{\vec{m}}\right)} \sums    
     \exp{\left(-\beta E(\vec{s}) + \beta\qhat{\vec{m}}\cdot\vec{s}\right)}\right]} \\
   \nonumber
   &= \log{\left[\sums\exp{\left(-\beta E(\vec{s}) +     
      \beta\qhat{\vec{m}}\cdot(\vec{s}-\vec{m})\right)}\right]} \\
   \label{eq:A-nicer-form}
   &= \log{\left[\sums\exp{\left(-\beta E(\vec{s}) +     
       \Lambd{\vec{m}}{\beta}\cdot(\vec{s}-\vec{m})\right)}\right]}
\end{align}
where we have defined the
\[
  \Lambd{\vec{m}}{\beta} \coloneqq \beta \qhat{\vec{m}}.
\]
\ExplainBox{The limit \(\beta \to 0\) for \(\vec{\lambda}\)}{
  Looking at its the definition one could naively think that \(\Lambd{\vec{m}}{0}=0\).
  This is not true because by expanding \eqref{eq:mean-magnetization} one can see that\footnote{
    I'm bluffing; I've never done the calculation, but I'm confident  that what I'm saying it's true.}
 \[
  \left|\lim_{\beta\to0} \qhat{\vec{m}}\right| \to+\infty,
 \]
 so \(\vec{\lambda}\) can take finite values at \(\beta = 0\):
 \[\Lambd{\vec{m}}{0}\neq0.\]
 \PhysicsBox{Why this makes sense physically}{
   The function \(\qhat{\vec{m}}\) is answering the question ``what must be the field for the magnetization to be equal to \(\vec{m}\)?''. The limit of \(\beta\to0\) correspond to high temperatures, so the system it's much more disordered by thermal fluctuations. If we want
   to keep the magnetization fixed (so impose some kind of order) we have to use a  strong field
   in order to overcome the effect of temperature. It's natural that the field diverges.
 }
}
\NotationBox{\Gmean{\cdot}{\vec{q}}{\beta}}{
  In analogy of the mean over \(F\):
  \[
  \gamma{\left[\vec{m};\beta\right](\vec{s})} \coloneqq \exp{\left(-\beta E(\vec{s}) +     
    \Lambd{\vec{m}}{\beta}\cdot(\vec{s}-\vec{m})\right)},
  \]
  and 
  \[
  \Gmean{O}{\vec{m}}{\beta} =
  \frac{\sums O{(\vec{s})}\gamma{\left[\vec{m};\beta\right](\vec{s})}}
  {\sum_{\vec{s'}}\gamma{\left[\vec{m};\beta\right](\vec{s'})}}.
  \]
  \PhysicsBox{What is this in physical term?}{
    It's just the mean value over the ensemble of the operator \(O\), written as
    function of independent variables \(\vec{m}\).
  }
}
\subsection[Expansion of A around \textbeta=0]{Expansion of \(A\) around \(\beta=0\)}
Deal with the exact form of \(A\) given by equation~\eqref{eq:A-nicer-form} is hard,
so we expand around \(\beta = 0\) and use the truncated series for further steps.
Of course since in the end we need the result for \(\beta=1\) more terms we keep, better
it works. In our derivation we use terms up to the third order.
\(A\) can be written as
\[
  A{\left[\vec{m};\beta\right]} \approx
    A{\left[\vec{m};0\right]} + 
    \beta \left.\ParDer{A{\left[\vec{m};\beta\right]}}{\beta}\right|_{\beta=0} +
    \frac{\beta^2}{2}
      \left.\frac{\partial^2A{\left[\vec{m};\beta\right]}}{\partial\beta^2}\right|_{\beta=0}+
    \frac{\beta^3}{6} 
      \left.\frac{\partial^3A{\left[\vec{m};\beta\right]}}{\partial\beta^3}\right|_{\beta=0}    
\]
Before going through the computation of the derivatives, we prove 3 useful lemmas.
\begin{lemma} \label{lem:lambda-t-zero}
  \(\Lambd{\vec{m}}{\beta}\) at \(\beta = 0\) is
  \[
    \lgtmz = \log{\left(m_t\right)} - \log{\left(1-m_t\right)}.
  \]
  \proof 
\end{lemma}
\begin{lemma} \label{lem:first-derivative-lambda}
  The derivative of \(\Lambd{\vec{m}}{\beta}\) at \(\beta = 0\) is
  \begin{align*}
    \left.\ParDer{\lvim{\beta}}{\beta}\right|_{\beta=0} &= -\sum_{j=1}^n w_{i,j}m^h_j-b_i\\
    \left.\ParDer{\lhjm{\beta}}{\beta}\right|_{\beta=0} &= -\sum_{i=1}^m w_{i,j}m^v_i-c_j
  \end{align*}
  \proof
\end{lemma}
\begin{lemma} \label{lem:operator-mean-gamma}
  Let \(O{(\vec{s})}\) an operator of the spins that can be written as a polynomial in the \(s_t\)
  variables, and \textbf{has at most degree 1 in each \(s_t\)}. Then
  \[
    \Gmean{O}{\vec{m}}{0} = O{(\vec{m})}.
  \]
  \ExplainBox{Why the hypotesis on the degree}{
    The key fact is that in our definition of RBM the nodes take values in \(\{0,1\}\), and both of these
    value has the property \(s_t^k = s_t \quad \forall k\in\mathbb{N}\), so when you write any power of
    a node value it's the same as writing the value itself. In example we 
    \[
      O{(\vec{s})} = 3s_1^2s_2 + 4s_1^3 + 5s_2^7 - 2 s_1 s_2^4 \equiv 4s_1+5s_2+s_1 s_2
    \]
    The energy instead is a proper operator on which apply the Lemma.
  }
  \proof
\end{lemma}
\subsubsection{Order 0}
The computation is easy
\begin{align*}
  A{\left[\vec{m};0\right]}
  &= \log{\left[\sums\exp{\left(\Lambd{\vec{m}}{0}\cdot(\vec{s}-\vec{m})\right)}\right]}
   = \log{\left[\sums\exp{\left(\sumt\lgtmz(s_t-m_t)\right)}\right]}\\
  &= \log{\left[\sums\prodt\exp{\left(\lgtmz(s_t-m_t)\right)}\right]}
   = \log{\left[\prodt\sumst\exp{\left(\lgtmz(s_t-m_t)\right)}\right]}\\
  &= \sumt\log{\left[e^{-\lgtmz m_t}(1+e^{\lgtmz})\right]}
   = \sumt\left[-\lgtmz m_t + \log{\left(1+e^{\lgtmz}\right)}\right]
\end{align*}
Now we can use the Lemma~\ref{lem:lambda-t-zero} and substitute \(\lgtmz\)
\begin{align*}
  A{\left[\vec{m};0\right]}
  &= \sumt\left[-\left(\log{\left(m_t\right)} - \log{\left(1-m_t\right)}\right) m_t
                + \log{\left(1+\frac{m_t}{1-m_t}\right)}\right] \\
  &= \sumt\left[-\log{\left(m_t\right)}m_t + \log{\left(1-m_t\right)}m_t
                - \log{\left(1-m_t\right)}\right] \\
  &= -\sumt\left[\log{\left(m_t\right)}m_t + \log{\left(1-m_t\right)}(1-m_t)\right] \\
  &= \sumt S_2{(m_t)}
\end{align*}
where \(S_2{(\cdot)}\) is the entropy of a binary distribution.
\PhysicsBox{Order 0 in physics}{
  The order 0 is the total entropy of the system when spins are not correlated.
}

\subsubsection{Order 1}
We compute the derivative starting from equation~\eqref{eq:A-nicer-form}
\begin{align*}
  \left.\ParDer{A{\left[\vec{m};\beta\right]}}{\beta}\right|_{\beta=0}
  &= \left.
    \frac{
      \sums\left(-E{(\vec{s})}+\ParDer{\Lambd{\beta}{\vec{m}}}{\beta}\cdot(\vec{s}-\vec{m})\right)\GammaExp{}
    }{
      \sum_{\vec{s'}}\GammaExp{'}
    }\right|_{\beta=0}\\
  &= \Gmean{-E{(\vec{s})}}{\vec{m}}{0} +
     \left.\ParDer{\Lambd{\beta}{\vec{m}}}{\beta}\right|_{\beta=0}\cdot\Gmean{\vec{s}-\vec{m}}{\vec{m}}{0}\\
  &= -E{(\vec{m})}
\end{align*}
In the last equality we have used the Lemma~\ref{lem:operator-mean-gamma}.
\PhysicsBox{Order 1 in physics}{
  The order 1 is the total energy of the system (multiplied by \(\beta\) where spins are setted equal
  to the mean magnetization.
}

\subsubsection{Order 2}
Starting again from equation~\eqref{eq:A-nicer-form} after little calculations we get
\begin{align*}
  \left.\frac{\partial^2A{\left[\vec{m};\beta\right]}}{\partial\beta^2}\right|_{\beta=0}
  &= \Gmean{
       \left.\frac{\partial^2\vec{\lambda}{\left[\vec{m};\beta\right]}}{\partial\beta^2}\right|_{\beta=0}
         \cdot (\vec{s}-\vec{m})
       +\left(-E{(\vec{s})} +   
          \left.\ParDer{\Lambd{\beta}{\vec{m}}}{\beta}\right|_{\beta=0}\cdot(\vec{s}-\vec{m})\right)^2
     }{\vec{m}}{0}+\\
  &\quad+ \Gmean{
          -E{(\vec{s})}+\left.\ParDer{\Lambd{\beta}{\vec{m}}}{\beta}\right|_{\beta=0}\cdot(\vec{s}-\vec{m})
       }{\vec{m}}{0}^2 \\
  &= \Gmean{
       \left(-E{(\vec{s})} +   
        \left.\ParDer{\Lambd{\beta}{\vec{m}}}{\beta}\right|_{\beta=0}\cdot(\vec{s}-\vec{m})\right)^2
     }{\vec{m}}{0} - \left[E{(\vec{m})}\right]^2
\end{align*}
Let's focus on the term inside the bracket: we can use Lemma~\ref{lem:first-derivative-lambda}
to evaluate the derivative of \(\vec{\lambda}\) (only in these few passage we use Einstein notation
for summed indexes)
\begin{align*}
  -E{(\vec{s})} + \left.\ParDer{\Lambd{\beta}{\vec{m}}}{\beta}\right|_{\beta=0}\cdot(\vec{s}-\vec{m})
  &= v_iw_{i,j}h_j +b_iv_i+c_jh_j - (v_i-\mvi)(w_{i,j}\mhj +b_i) - (\mvi w_{i,j}+c_j)(h_j-\mhj) \\
  &= v_iw_{i,j}h_j - v_i w_{i,j} \mhj - \mvi w_{i,j} h_j + \mvi w_{i,j} \mhj - E{(\vec{m})}\\
  &= w_{i,j} (v_i-\mvi)(h_j-\mhj) - E{(\vec{m})}
\end{align*}
We can substitute this result in the expression above
\begin{align*}
  &\left.\frac{\partial^2A{\left[\vec{m};\beta\right]}}{\partial\beta^2}\right|_{\beta=0}=\\
  &\qquad= \Gmean{
      \sum_{i,j=1}^{m,n}\sum_{p,q=1}^{m,n} w_{i,j} w_{p,q}(v_i-\mvi)(h_j-\mhj)(v_p-m_p^v)(h_q-m_q^h)
      +2E{(\vec{m})}\sum_{i,j=1}^{m,n}w_{i,j}(v_i-\mvi)(h_j-\mhj)
    }{\vec{m}}{0}
\end{align*}
As direct consequence of Lemma~\ref{lem:operator-mean-gamma} if a term is proportional to \(s_t-m_t\)
it cancels out when averaged on \(\gamma\). This fact makes the second term of the expression go away,
and in the first only the ones with \(i=p\) and \(j=q\) survive
\begin{align*}
  \left.\frac{\partial^2A{\left[\vec{m};\beta\right]}}{\partial\beta^2}\right|_{\beta=0}
    &=\Gmean{\sum_{i,j=1}^{m,n} w_{i,j}^2\left(v_i-\mvi\right)^2\left(h_j-\mhj\right)^2}{\vec{m}}{0} \\
    &=\Gmean{\sum_{i,j=1}^{m,n} w_{i,j}^2\left(v_i^2 -2v_i\mvi+\mvisq\right)\left(h_j^2-2h_j\mhj +\mhjsq\right)}{\vec{m}}{0}\\
    &\equiv\Gmean{\sum_{i,j=1}^{m,n} w_{i,j}^2\left(v_i-2v_i\mvi+\mvisq\right)\left(h_j-2h_j\mhj +\mhjsq\right)}{\vec{m}}{0}\\
    &=\sum_{i,j=1}^{m,n} w_{i,j}^2\left(\mvi-\mvisq\right)\left(\mhj-\mhjsq\right)
\end{align*}
where we used the equivalence \(s_t^2 \equiv s_t\), and the Lemma~\ref{lem:operator-mean-gamma}
for computing the average.

\subsubsection{Order 3}
The calculation of the third order is not too different from the second order, it only need more
steps and an expression for the second derivative of \(\vec{\lambda}\) that can be derived easily.
We don't report the full calculation, but only the result taken from \cite{gabrie18training}
\[
  \left.\frac{\partial^3A{\left[\vec{m};\beta\right]}}{\partial\beta^3}\right|_{\beta=0}
  = 4 \sum_{i,j=1}^{m,n} w_{i,j}^3\left(\mvi-\mvisq\right) \left(\frac{1}{2}-\mvi\right)
                                  \left(\frac{1}{2}-\mhj\right) \left(\mhj-\mhjsq\right).
\]
Alternatively this result (and also higher and lower orders) can be computed using the formalism
developed in \cite{georges1991expand} which makes the calculations much more easier.
\ExplainBox{Why don't we use order higher than 3}{
  The reason is because computing higher order is so computationally expensive that it is
  not worth it. The \(k\)-th order can be thought as the sum over the \emph{fully connected
  sub-graph} of the RBM, with \emph{\(k\) arcs} and where edges can be considered multiple
  times\footnote{
    This picture does not gives coefficients (that still require a full calculation to be computed),
     but it only help to double-check the calculations.
  }.
  In this picture the order we have computed are
  \begin{itemize}
    \item order 0 are all the single nodes;
    \item order 1 are all the pairs directly connected, with the corresponding edge;
    \item order 2 are again all the pairs directly connected, but with the corresponding edge
      considered 2 times;
    \item order 3 can have 2 different topologies \emph{a priori}:
      \begin{itemize}
        \item pairs directly connected with the corresponding edge considered 3 times;
        \item triangles;
      \end{itemize}
      But triangles are not allowed by the RBM's topology, since there aren't edges between
      nodes in the same layer.
  \end{itemize}
  So to compute the first 3 orders we have to sum only on the edges, that it is still reasonable.
  The fourth order instead (and also orders above) has a \emph{quadrilateral} topology that does not
  go away, so it requires a sum on all possible pair of edges, that is too much.
  
  In \cite{georges1991expand} there are some drawings that can help in understanding the picture
  we have described.
}

\subsubsection{Put all the pieces together}
Now we have computed all the orders that we need we can write
\begin{align*}
A{\left[\vec{m}^v, \vec{m}^h;\beta\right]} &\approx
  -\sum_{i=1}^{m}\left[\log{\left(\mvi\right)}\mvi + \log{\left(1-\mvi\right)}(1-\mvi)\right]
  -\sum_{j=1}^{n}\left[\log{\left(\mhj\right)}\mhj + \log{\left(1-\mhj\right)}(1-\mhj)\right]\\
  &\quad-\beta \sum_{i,j=1}^{m,n}\mvi w_{i,j} \mhj -\beta\sum_{i=1}^{m} b_i\mvi -\beta\sum_{j=1}^{m} c_j\mhj\\
  &\quad+\frac{\beta^2}{2}
  \sum_{i,j=1}^{m,n} w_{i,j}^2\left(\mvi-\mvisq\right)\left(\mhj-\mhjsq\right)\\
  &\quad+\frac{2}{3}\beta^3 
    \sum_{i,j=1}^{m,n} w_{i,j}^3\left(\mvi-\mvisq\right) \left(\frac{1}{2}-\mvi\right)
    \left(\frac{1}{2}-\mhj\right) \left(\mhj-\mhjsq\right),
\end{align*}
that is the approximated formula we will use to do our further calculations.

\subsection{Estimation of the log-likelihood gradient}
The quantity we are interested is \(\FEn{\vec{0}}{1}\) that can be calculated by minimizing \(\Gam{\vec{m}}{1}\).
First of all we set \(\beta=1\) since as in the RBM distributions, and using the approximation
on \(A\) we write
\begin{align*}
  \Gam{\vec{m}^v, \vec{m}^h}{1} &\approx
    \quad\sum_{i=1}^{m}\left[\log{\left(\mvi\right)}\mvi + \log{\left(1-\mvi\right)}(1-\mvi)-b_i\mvi\right]\\
    &\quad+\sum_{j=1}^{n}\left[\log{\left(\mhj\right)}\mhj+\log{\left(1-\mhj\right)}(1-\mhj)-c_j\mhj\right]\\
    &\quad+\sum_{i,j=1}^{m,n}\Bigg[
      -\mvi w_{i,j}\mhj
      +\frac{1}{2}w_{i,j}^2\left(\mvi-\mvisq\right)\left(\mhj-\mhjsq\right)+\\
      &\qquad\qquad\quad+\frac{2}{3}w_{i,j}^3\left(\mvi-\mvisq\right)\left(\frac{1}{2}-\mvi\right)\left(\frac{1}{2}-\mhj\right)
        \left(\mhj-\mhjsq\right)
    \Bigg]
\end{align*}
We use this equation both for finding the minimum \(\vec{\hat{m}}\), and the to compute the gradient.

\subsubsection{Equation for the minimum of \(\Gamma\)}
Since \(\Gamma\) is convex and smooth we have an easy equation for \(\vec{\hat{m}}\)
\[
  \left.\ParDer{\Gam{\vec{m}}{1}}{\vec{m}}\right|_{\vec{m}=\vec{\hat{m}}}=0.
\]
Let's compute the gradient respect to \(\mvi\), the case \(\mhj\) is  the same.
\begin{align*}
  \hspace*{-1.2cm}
  \left.\ParDer{\Gam{\vec{m}}{1}}{\vec{m}}\right|_{\vec{m}=\vec{\hat{m}}}
    &= \log\left(\hmvi\right) + 1 - \log\left(1-\hmvi\right) - 1 - b_i \\
    &\quad-\sum_{j=1}^n\left[w_{i,j}\hmhj
      +\frac{1}{2}w_{i,j}\left(1-2\hmvi\right)\left(\hmhj-\hmhjsq\right)
      +\frac{2}{3}w_{i,j}^3\left(3\hmvisq-3\hmvi+\frac{1}{2}\right)\left(\frac{1}{2}-\hmhj\right)
      \left(\hmhj-\hmhjsq\right)\right].
\end{align*}
Now we notice that
\[
  \log{\left(\frac{y}{1-y}\right)} = x \quad \implies \quad y = \sigmoid{x},
\]
and  applying to the equation above we get the \emph{consistency relations} for the magnetization
\begin{align}
  \hmvi&=\sigmoid{b_i + \sum_{j=1}^n\left[w_{i,j}\hmhj+
           \left(\hmhj-\hmhjsq\right)\left(w_{i,j}\left(\frac{1}{2}-\hmvi\right)
           +w_{i,j}^3\left(2\hmvisq-2\hmvi+\frac{1}{3}\right)\left(\frac{1}{2}-\hmhj\right)
           \right)\right]}\\
  \hmhj&=\sigmoid{c_j + \sum_{i=1}^m\left[w_{i,j}\hmvi+
           \left(\hmvi-\hmvisq\right)\left(w_{i,j}\left(\frac{1}{2}-\hmhj\right)
           +w_{i,j}^3\left(2\hmhjsq-2\hmhj+\frac{1}{3}\right)\left(\frac{1}{2}-\hmvi\right)
           \right)\right]}
\end{align}
Solving these equations give us \(\vec{\hat{m}}\) to be used when evaluating the gradient.  
